\section{Introduction}
\label{ch:introduction}

\subsection{Background of the Study}
The Philippines, despite being an agricultural nation, faces significant challenges in post-harvest management. High-value crops like tomatoes (\textit{Solanum lycopersicum}) suffer post-harvest losses estimated at 20--40\%, primarily due to poor handling, lack of cold chain infrastructure, and inadequate storage facilities \cite{fao2019}. Unlike grain crops, tomatoes are climacteric fruits that continue to respire and ripen after harvest, making temperature control critical for extending shelf life and ensuring market viability.

Commercial operations mitigate these losses using industrial ripening chambers equipped with precise climate control systems. These facilities use ethylene gas injection and active refrigeration to synchronize ripening, ensuring uniform quality for supermarkets. However, the capital expenditure for such infrastructure ranges from \$10,000 to \$50,000 \cite{prasad2018}, effectively excluding the 5.5 million smallholder farming households in the Philippines who typically earn less than \$2,000 annually. As a result, small farmers are forced to sell their produce immediately after harvest at fluctuating farm-gate prices, often leading to income instability and food waste.

\subsection{Statement of the Problem}
Smallholder tomato farmers lack access to affordable, intelligent post-harvest decision-support systems. Existing solutions fall into two extremes: passive monitoring tools (IoT sensors) that provide data but no actionable control, and high-end industrial systems that are cost-prohibitive. Cloud-based Reinforcement Learning (RL) solutions have shown promise in controlled environment agriculture but require continuous internet connectivity, which is unreliable or absent in many rural Philippine farming communities. There is currently no standalone, low-cost system capable of autonomous, optimal ripening control at the edge.

\subsection{Objectives of the Study}
This study aims to develop "Edge-RL," a standalone autonomous ripening control system running on a low-cost microcontroller.

\subsubsection{General Objective}
To develop a low-cost, offline Reinforcement Learning system on an ESP32-S3 microcontroller that optimizes the post-harvest ripening of tomatoes by balancing quality preservation with energy efficiency.

\subsubsection{Specific Objectives}
\begin{enumerate}
    \item To design a low-cost, standalone ripening chamber hardware (BOM $<$ \$50) integrating an ESP32-S3, camera, and relay-controlled heating element.
    \item To implement a Deep Q-Network (DQN) policy trained in a physics-based digital twin that generalizes across tomato cultivars using domain randomization.
    \item To validate the system's performance by distilling the policy to an edge-optimized model (INT8) and achieving a harvest timing error of less than 20\% in sim-to-real transfer.
\end{enumerate}

\subsection{Significance of the Study}

This study addresses the critical issue of post-harvest loss, which accounts for up to 42\% of fruit and vegetable production globally according to the FAO \cite{fao2019}. In developing economies like the Philippines, these losses are exacerbated by the lack of cold chain infrastructure, often forcing smallholder farmers to sell produce at low prices or face total spoilage \cite{worldbank2020}.

By developing a low-cost, intelligent ripening chamber, this research directly benefits:
\begin{itemize}
    \item \textbf{Smallholder Farmers:} Providing a tool to control the timing of produce sales, decoupling harvest time from market saturation.
    \item \textbf{Agricultural Logistics:} Reducing spoilage during transport \cite{kader2005} through precise biological control.
    \item \textbf{Technological Advancement:} Demonstrating the feasibility of deploying advanced Reinforcement Learning on \$2 microcontrollers (Edge AI) for complex biological systems \cite{warden2019tinyml}.
\end{itemize}

\subsection{Scope and Limitations}

\subsubsection{Scope}
This thesis encompasses the complete pipeline from simulation-based RL training to on-device inference on embedded hardware. The following boundaries define the system's scope:

\begin{enumerate}
    \item \textbf{Single-Fruit Focus:} The system monitors and controls the ripening of a single tomato unit within an enclosed chamber. This isolates the ripening kinetics of one fruit and avoids the complexity of batch effects such as inter-fruit ethylene signaling.

    \item \textbf{Heater-Only Actuation:} The control mechanism is limited to a resistive heating element (raising temperature above ambient) and passive ventilation (cooling toward ambient). No active refrigeration (compressor or Peltier element) is used, constraining the controllable temperature range to $T_{\text{ambient}} \le T_{\text{chamber}} \le 35^\circ$C.

    \item \textbf{Target Crop:} The ripening model is calibrated for Philippine tomato varieties, specifically the commercial hybrid ``Diamante Max F1'' and native ``Kamatis Tagalog.'' The ripening rate constant $k_1$ is parameterized to reflect the kinetics of these cultivars.

    \item \textbf{Offline Edge Deployment:} All inference and control logic execute locally on the ESP32-S3 microcontroller. No cloud connectivity is required during operation; the system is designed for rural areas without reliable internet access.

    \item \textbf{Sim-to-Edge Pipeline:} The study validates the complete pipeline from Gymnasium-based environment simulation, through DQN teacher training and student distillation, to pure-C inference on the Xtensa LX7 core.

    \item \textbf{State-Space Ablation:} Three observation variants are evaluated (7D scalar, 16D with RGB statistics, 20D with spatial pooling) to determine the optimal state representation for edge deployment.
\end{enumerate}

\subsubsection{Limitations}
The following limitations constrain the generalizability and completeness of the current work:

\begin{enumerate}
    \item \textbf{Simulation-Only Training:} The RL policy is trained entirely in a physics-based digital twin. While domain randomization is applied to improve robustness, the policy has not been validated against real physical tomatoes in a closed-loop setting.

    \item \textbf{Simulated Colour Statistics:} The RGB colour statistics (mean, standard deviation, mode) used in Variant B and C are generated synthetically by the simulator. Real camera-derived statistics may exhibit different noise characteristics, lens distortion, and lighting dependencies.

    \item \textbf{No Active Cooling:} The absence of a compressor or Peltier module means the system cannot cool below ambient temperature. In tropical climates where $T_{\text{ambient}}$ may exceed 30$^\circ$C, the agent's ability to slow ripening is fundamentally limited to the MAINTAIN action.

    \item \textbf{No Ethylene Sensing:} The system does not incorporate ethylene gas sensors, which provide a direct biochemical indicator of climacteric ripening onset. Ripeness estimation relies solely on visual (RGB) and environmental (temperature, humidity) features.

    \item \textbf{Fixed Policy Architecture:} The student MLP architecture (64$\times$64, ReLU) is fixed at compile time. On-device fine-tuning or continual learning is not supported in the current firmware; the policy is static after deployment.

    \item \textbf{Single-Cultivar ODE:} The ripening ODE uses a single $k_1$ constant per episode. Real-world cultivar variability within a batch is not captured, and the model does not account for fruit maturity at harvest or mechanical damage.
\end{enumerate}


\subsection{Definition of Terms}
\begin{description}
    \item[Edge-RL] The proposed system architecture where Reinforcement Learning inference occurs directly on the edge device (microcontroller) rather than in the cloud.
    \item[Climacteric Fruit] Fruits that exhibit a spike in respiration and ethylene production during ripening (e.g., tomato, banana, mango).
    \item[Knowledge Distillation] A compression technique where a small "student" model learns to mimic the output of a larger "teacher" model.
    \item[Continuous Chromatic Index ($X$)] A computed scalar value derived from the spectral reflectance of the fruit, representing its ripeness stage on a continuous scale from Green ($X=1$) to Red ($X=0$).
\end{description}
