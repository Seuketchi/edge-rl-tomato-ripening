\section{Conclusion and Recommendations}
\label{sec:conclusion}

\subsection{Conclusion}
This study successfully demonstrated \textit{Edge-RL}, a novel framework for autonomous post-harvest ripening control deployed entirely on a standard ESP32-S3 microcontroller. By integrating INT8 quantization with policy distillation, we bridged the gap between complex reinforcement learning control and resource-constrained edge hardware.

Addressing our research objectives:
\begin{enumerate}
    \item \textbf{Feasibility (RQ1):} We confirmed that a complex DQN policy ($\sim$270\,KB) can be compressed into a 35\,KB INT8 MLP with $<$1\,ms inference time and $>$98\% behavioral fidelity, proving that sequential decision-making is viable on sub-\$10 MCUs.
    \item \textbf{Sim-to-Real Transfer (RQ2):} The physics-based digital twin with domain randomization effectively trained a robust policy that generalized to real-world sensor noise and cultivar variability without requiring physical retraining.
    \item \textbf{Operational Viability (RQ3):} The system outperformed heuristic baselines in quality preservation (+16\% reward) and eliminated spoilage entirely, validating the economic potential for smallholder farmers.
\end{enumerate}

The system provides a cost-effective ($<$ \$33), offline-capable alternative to expensive commercial ripening facilities, democratizing access to precision agriculture technology.

\subsection{Recommendations}
Based on the findings and limitations of this study, we recommend the following for future development:

\begin{enumerate}
    \item \textbf{Hardware Refinement:} Given the emergent diurnal strategy that actively utilizes both heating and cooling to regulate the biological Arrhenius kinetics, future iterations of the physical chamber must ensure the passive ventilation system provides sufficient airflow to achieve the rapid -1$^{\circ}$C cooling drops demanded by the policy. If passive ventilation proves insufficient in extreme tropical climates, integrating a low-power active Peltier cooling element may be required to maintain the precise 0.67-day timing accuracy observed in simulation.
    \item \textbf{Continuous Learning:} Implement on-device fine-tuning to allow the agent to adapt to specific local conditions or new tomato varieties over time, rather than relying solely on the pre-trained static policy.
    \item \textbf{Multi-Modal Sensing:} Integrate ethylene gas sensors to provide a direct biochemical indicator of ripening, potentially improving the state estimation accuracy beyond visual and environmental data alone.
    \item \textbf{Field Deployment:} Conduct extended pilot testing in actual farm environments to evaluate long-term hardware durability and the system's impact on farmer income and post-harvest loss reduction at scale.
\end{enumerate}
